\chapter{Conclusion}\label{ch:conclusion}
The purpose of this report was to investigate the effects from crosstalk at antenna arrays connected to PAs, and find a simple method to compensate for the imperfections. In chapter \ref{ch:3} some DPD models was presented together with a mathematical description of crosstalk. Is was shown that other has made a model for the crosstalk by the use of S-parameters and feedback from the amplifiers which tends to become rather comprehensive. Therfore a simple model was proposed and measurements in chapter \ref{ch:measurement} showed that this model could be useful. Futher the measurements showed that there was no problem when introducing one Tx antenna and one Rx antenna to a PA without compensating for the antennas. When introducing two amplifiers and two Tx antennas, then the crosstalk between the Tx antennas introduced distortion. It was seen that the distortion nearly was a constant value in its ACPR independent of the spacing between the antenna elements in the array. The same was seen for the case with four amplifiers and four Tx antennas. The measurement in section \ref{ch:ant_meas} of the antennas showed a large variation in the S-parameters due to the spacing of the antenna elements. It was therefore not expected that the ACPR in section \ref{ch:measurement} had no impact on the spacing. It was also measured if it had any impact that the amplifiers was biased with same bias voltage but different current due to transistor deviations. It was showed that it had little or no impact on the ACPR when treating the amplifiers with antennas as a hole.     



