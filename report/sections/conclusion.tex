\chapter{Conclusion}\label{ch:conclusion}
The purpose of this project was to investigate the effects from crosstalk at antenna arrays connected to PAs when they are linearised with DPD. It was shown that it is not necessary to include a mathematical model for the antenna if only one antenna is used at a single amplifier. It was further shown that when introducing two amplifiers with one antenna each the conventional DPD technique was not able to compensate for the crosstalk in the antennas. It was therefore tested to measure the setup as a hole and do DPD with the antennas included into the DPD model. This provided better results with a ACPR about 2dB better. When four amplifiers with one antenna each was introduced the technique still lowers the ACPR with 2dB over all distances between the elements in the antenna array. It was also seen from the AM/AM plots that the DPD model was overcompensating at high power-levels. Therfore the proposed model in figure \ref{fig:dpd_pdpd} is shown to be better than conventional DPD, but that there still are space left to improvements. The measurement in section \ref{ch:ant_meas} of the antennas showed a large variation in the S-parameters due to the spacing of the antenna elements. It was also measured if it had any impact that the amplifiers was biased with same bias voltage but different current due to transistor deviations. It was showed that it had little or no impact on the ACPR when treating the amplifiers with antennas as a hole.     



