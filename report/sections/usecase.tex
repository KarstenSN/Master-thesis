\chapter{Usecase}\label{ch:usecase}

Automatic dependent surveillance-broadcast (ADS-B) is a system in which aircraft continually transmit their identity and GPS-derived navigational information. ADS-B networks for air traffic monitoring have already been implemented in areas around the world, but ground stations cannot be installed in mid-ocean and are difficult to maintain in the Arctic, leaving a coverage gap for oceanic and high latitude airspace \citep{FlyingLab}. Therefore a solution can be to monitor the signals with a low orbit satellite using an antenna matched to the frequencies of the ADS-B.There are currently three types of ADS-B transmissions, including the 1090 MHz extended squitter (ES), the 978 MHz universal access transceiver (UAT), and the VHF data link (VDL) mode 4 operating between 108 and 137 MHz.Requirements for an satellite receiving antenna is listed below.
\begin{itemize}
  \item Cover the frequency ranges from 108-137MHz and 978-1090MHz
  \item To be stowed in a 1U cubesat before and on launch
  \item Unfold when in orbit
  \item Circular polarized
\end{itemize}