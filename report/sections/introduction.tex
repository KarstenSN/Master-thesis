\chapter{Introduction}\label{ch:introduction}

This master thesis is aiming at the linearization of Power amplifiers with the presence of cross talk in antenna arrays. The tasks includes the characterization of the cross talk and its impact on the system linearity and digital pre-distrotion of Power Amplifiers (PA) under effects of antenna cross talks. The project is motivated by the quickly growing need from the mobile communication industry, where highly integrated beam-steerable arrays consisting of a large number of power amplifiers and antenna elements are considered as the solution for higher data rates desired in emerging applications such as self-driving cars, remote e-health etc. There will be use Digital Pre Distortion (DPD) techniques that can capture the total non-linearity of the whole array including the PAs and antenna elements. The special focus is to reduce the system complexity, while maintaining the linearisation performance. This project will be conducted in a way combining mainly measurements together with computaion in MATLAB.
\\
\\
The report is divided into 5 sections where the first is this introduction. The second is some basic theory that gives the reader a basic understanding of the measurement that has been done in section 4. Section 3 describes how to mathematically describe a PA and how to linearise it with DPD, a description of crosstalk is also given together with a simple method to capture the non-linearity of the system. To test this model measurement has been provided in section 4 where measurement has been done at a single PA directly connected to a spectrum analyzer. Also two PAs with one antenna each, and four PAs with one antenna each has been measured with a receiver antenna at the spectrum analyzer. Section 5 presents measurement of the farfield and S-parameters of the used ansennas to explain the results from section 4.    


