\chapter{Measurement}\label{ch:measurement}
The purpose of this section is to characterize a power amplifier for its distortion mechanisms and then measure the impact of the crosstalk from the antennas. 

\section{Simulation of PCB antenna}

\section{Power Amplifier}
The  \textit{ZX60-6013E+} amplifier is a small buffer-amplifier with a BW from 20MHz - 6GHz. It has an output-power at maximum 11.16dBm at 3.5GHz (1 dB compression ). The gain is 13.85 dB and the noise figure is 3.42dB. In figure \ref{fig:amp_gain} and \ref{fig:amp_outpower} curves for the amplifier gain and output power versus frequency it depicted respectively. The amplifier is shown in figure \ref{fig:amplifier} 

\begin{figure}[H]
\centering 
\includegraphics[scale = 0.3]{figures/measurement/amplifier.png}
\caption{The \textit{ZX60-6013E+} amplifier}
\label{fig:amplifier}
\end{figure} 

\begin{figure}[H]
  \centering
  \begin{minipage}[b]{0.5\textwidth}
	\includegraphics[scale = 0.7]{figures/measurement/amp_gain.png}
	\caption{Gain versus frequency}
    \label{fig:amp_gain}
  \end{minipage}
  \hfill
  \begin{minipage}[b]{0.4\textwidth}
\includegraphics[scale = 0.7]{figures/measurement/amp_outpower.png}
\caption{Output power versus frequency}
    \label{fig:amp_outpower}
  \end{minipage}
\end{figure}


\section{Measurement setup} \label{ch_meas_setup}
To measure the impact of the antenna crosstalk, several measurements must be done at different setup. The first setup is shown in figure \ref{fig:Meas_setup1}. In this setup only the amplifier is measured because it is necessary with a reference. The signal generator is generating a 10MHz LTE signal at 3.5GHz which output is then connected to the input of the amplifier. The amplifier is supplied with a 12.0VDC signal from the power supply and the output of the amplifier is connected to a 6dB attenuator to protect the input stage of the spectrum analyser, which measures the output signal from the amplifier.   

\begin{figure}[H]
\centering 
\includegraphics[scale = 0.6]{figures/measurement/meas_set_1.png}
\caption{Block diagram of measurement at amplifier }
\label{fig:Meas_setup1}
\end{figure} 

In figure \ref{fig:Meas_setup2} the second measurement setup is shown. In this setup the antennas are now introduced. The Tx antenna is connected to the output of the amplifier. The Rx antenna are spaced 1 meter apart from the Tx antenna (measured from feed to feed). The 6dB attenuator is removed and the antenna is connected to the spectrum analyzer. 


\begin{figure}[H]
\centering 
\includegraphics[scale = 0.6]{figures/measurement/meas_set_2.png}
\caption{Block diagram of measurement using one antenna}
\label{fig:Meas_setup2}
\end{figure} 

In setup three a power-divider is now introduced together with a second amplifier and another antenna. The received power will now increase 3dB but because the amplifiers has to be driven in the same power levels as before, one must add another 3dB to the input of the power divider. Be aware that in the measurements a 1:4 power divider was used and therefore the power was increased 6dB. 

\begin{figure}[H]
\centering 
\includegraphics[scale = 0.6]{figures/measurement/meas_set_3.png}
\caption{Block diagram of measurement using two antennas}
\label{fig:Meas_setup3}
\end{figure} 


In figure \ref{fig:Meas_setup} a picture of a real measurement setup is shown. The antennas are spaced 1 meter apart from the input terminals and are pointing directly towards each other. The PA is mounted on a aluminium sheet with a tool grip to keep it at a constant temperature. The PA is supplied with 12.0VDC from the power supply and draws 40mA of current. The signal generator is connected to the input of the amplifier and the output of the amplifier is connected to the Tx antenna. The spectrum analyser is connected to Rx antenna. 


\begin{figure}[H]
\centering 
\includegraphics[scale = 0.1]{figures/measurement/measurement_setup.jpg}
\caption{Measurement setup}
\label{fig:Meas_setup}
\end{figure} 




\subsection{Measurement of AM/AM distortion}\label{ch:meas_amam}
Measurement has been done using the methods described in section \ref{ch_meas_setup}. The results in figure \ref{fig:amam_amp} is a measurement of only the amplifier. It is seen that at an input of -2.5dB the amplifier starts to decrese its gain. This is expected due to the compression point. The mean gain of the amplifier is  measured closly to 13.8dB as expected. It is further seen that the spreading of the signal is highest at low signal levels. When introducing one Tx antenna and one Rx antenna the free space loss must be accounted for. This is done by use of equation \ref{eq:friss_1m} which gives 20.5dB in loss. In the measurement 19.5dB was obtained, the diffrence is might caused by the antenna them self, since the used antenna gain is only simulated, and the antenna are spaced relativity close, or it can be reflections from the table that are causing the difference. Cable loss and connector loss has been measured and accounted for.  

\begin{equation}\label{eq:friss_1m}
Pathloss(db) = 10*log10(G_r G_t (\frac{4\pi d}{\lambda})^2)=20.5dB
\end{equation}

The measurement using one Tx antenna and one Rx antenna shows that the gain of the system now becomes close to 6dB, this is also expected since the gain of the system is: 

\begin{equation}
G_{system}(dB) = G_{amplifier}+G_{antenna}-Pathloss = 13.85dB + 11.4dB -19.5dB = 5.75dB
\end{equation}

It can be seen from figure \ref{fig:amam_one_ant} that the antenna system makes the AM/AM distortion wider at the compression point, but smaller at lower power levels. In figure \ref{fig:amam01} to \ref{fig:amam06} the results from measurement setup three is shown. The results are made with two transmit antennas and therefore the system gain is increased by 3dB. The results shows small variations due to AM/AM distortion between, but a significant change in AM/AM at compression point in the pre-sentence of two antennas instead of one. Also AM/AM at lower power levels seems to decrease with two antennas compared to one. 

\begin{figure}[H]
  \centering
  \begin{minipage}[b]{0.5\textwidth}
	\includegraphics[scale = 0.5]{figures/measurement/two_antenna/amplifier_amam.png}
	\caption{AM/AM distortion at amplifier}
    \label{fig:amam_amp}
  \end{minipage}
  \hfill
  \begin{minipage}[b]{0.4\textwidth}
\includegraphics[scale = 0.5]{figures/measurement/two_antenna/one_ant_amam.png}
\caption{AM/AM distortion using one transmit antenna}
    \label{fig:amam_one_ant}
  \end{minipage}
\end{figure}


\begin{figure}[H]
  \centering
  \begin{minipage}[b]{0.5\textwidth}
	\includegraphics[scale = 0.5]{figures/measurement/two_antenna/amam_01.png}
	\caption{AM/AM distortion at 0.1 $\lambda$}
    \label{fig:amam01}
  \end{minipage}
  \hfill
  \begin{minipage}[b]{0.4\textwidth}
\includegraphics[scale = 0.5]{figures/measurement/two_antenna/amam_02.png}
\caption{AM/AM distortion at 0.2 $\lambda$}
    \label{fig:amam02}
  \end{minipage}
\end{figure}

\begin{figure}[H]
  \centering
  \begin{minipage}[b]{0.5\textwidth}
	\includegraphics[scale = 0.5]{figures/measurement/two_antenna/amam_03.png}
	\caption{AM/AM distortion at 0.3 $\lambda$}
    \label{fig:amam03}
  \end{minipage}
  \hfill
  \begin{minipage}[b]{0.4\textwidth}
\includegraphics[scale = 0.5]{figures/measurement/two_antenna/amam_04.png}
\caption{AM/AM distortion at 0.4 $\lambda$}
    \label{fig:amam04}
  \end{minipage}
\end{figure}

\begin{figure}[H]
  \centering
  \begin{minipage}[b]{0.5\textwidth}
	\includegraphics[scale = 0.5]{figures/measurement/two_antenna/amam_05.png}
	\caption{AM/AM distortion at 0.4 $\lambda$}
    \label{fig:amam05}
  \end{minipage}
  \hfill
  \begin{minipage}[b]{0.4\textwidth}
\includegraphics[scale = 0.5]{figures/measurement/two_antenna/amam_06.png}
\caption{AM/AM distortion at 0.5 $\lambda$}
    \label{fig:amam06}
  \end{minipage}
\end{figure}

%%%%%%%%%%%%%%%%%%%%%%%%%%%%%%%%%%%%%%%%%%%%%%%%%%%%%%%%%%%%%%%%%%%%%%%%%%%%%%%%%%%%%%%%%%%%%%%%%%%%%%%%%%%%%%%%5
\subsection{PSD}
The Power Sprectal Density (PSD) shown in figure \ref{fig:psd_amp} is measured directly at the amplifier. It shows that at a mean input-power at 0dBm the output of the amplifier is distorted, first at a input level at -9dBm the distortion becomes low. This is also expected since the input is mean power and therefore the peak of the signal would be closely to -3dBm (6dB backoff) which also is close to the compression point measured in section \ref{ch:meas_amam}. Be aware that the noise floor increases due to the normalization of the signal. Also the reference-level at the spectrum analyser has an impact. When introducing the antenna system with one Tx antenna the distortion at -9dBm increases slightly. When introducing two antennas the distortion increases which is not a surprise due to the results from section \ref{ch:meas_amam}.  


\begin{figure}[H]
  \centering
  \begin{minipage}[b]{0.5\textwidth}
	\includegraphics[scale = 0.5]{figures/measurement/two_antenna/amplifier_psd.png}
	\caption{PSD at amplifier}
    \label{fig:psd_amp}
  \end{minipage}
  \hfill
  \begin{minipage}[b]{0.4\textwidth}
\includegraphics[scale = 0.5]{figures/measurement/two_antenna/one_ant_psd.png}
\caption{PSD using one transmit antenna}
    \label{fig:psd_one_ant}
  \end{minipage}
\end{figure}

\begin{figure}[H]
  \centering
  \begin{minipage}[b]{0.5\textwidth}
	\includegraphics[scale = 0.5]{figures/measurement/two_antenna/psd_01.png}
	\caption{PSD at 0.1 $\lambda$}
    \label{fig:psd01}
  \end{minipage}
  \hfill
  \begin{minipage}[b]{0.4\textwidth}
\includegraphics[scale = 0.5]{figures/measurement/two_antenna/psd_02.png}
\caption{PSD at 0.2 $\lambda$}
    \label{fig:psd02}
  \end{minipage}
\end{figure}

\begin{figure}[H]
  \centering
  \begin{minipage}[b]{0.5\textwidth}
	\includegraphics[scale = 0.5]{figures/measurement/two_antenna/psd_03.png}
	\caption{PSD at 0.3 $\lambda$}
    \label{fig:psd03}
  \end{minipage}
  \hfill
  \begin{minipage}[b]{0.4\textwidth}
\includegraphics[scale = 0.5]{figures/measurement/two_antenna/psd_04.png}
\caption{PSD at 0.4 $\lambda$}
    \label{fig:psd04}
  \end{minipage}
\end{figure}

\begin{figure}[H]
  \centering
  \begin{minipage}[b]{0.5\textwidth}
	\includegraphics[scale = 0.5]{figures/measurement/two_antenna/psd_05.png}
	\caption{PSD at 0.5 $\lambda$}
    \label{fig:psd05}
  \end{minipage}
  \hfill
  \begin{minipage}[b]{0.4\textwidth}
\includegraphics[scale = 0.5]{figures/measurement/two_antenna/psd_06.png}
\caption{PSD at 0.6 $\lambda$}
    \label{fig:psd06}
  \end{minipage}
\end{figure}

\subsection{ACPR} %%%%%%%%%%%%%%%%%%%%%%%%%%%%%%%%%%%%%%%%%%%%%%%%%%%%%%%%%%%%%%%%%%%%%%%%%%%%%%%%%%%%%%%%%%%%%%%

\begin{figure}[H]
\centering 
\includegraphics[scale = 1]{figures/measurement/two_antenna/ACPR.png}
\caption{ACPR}
\label{fig:acpr_meas}
\end{figure} 


