\chapter{MATLAB code}\label{ch:appAlabel}

Code to plot AM/AM distortion. powerLev is peak power level at signel generator, preGain is gain of pre-amplifier, Att is attenuation at spectrum analyzer. 

\begin{lstlisting}[frame = single]
powerLev = -11;
preGain = 44;
Att = 31;

input =  10*log10(abs(e0)./max(abs(e0)))+preGain+powerLev;  

for i=1:N
    figure;
    gain = 20*log10(abs(e(i,:))*10^(Att/20)./
    (abs(e0)*10^(preGain/20)))-(powerLev);
    plot(input,gain-6,'.r');
    grid on
    title('AM/AM')
    xlabel('Peak input power (dBm)')
    ylabel('Gain of amplifier (dB)')
    legend('With sys DPD')
    ylim([-6 4])
    xlim([10 28])
end

\end{lstlisting}


Code to plot PSD

\begin{lstlisting}[frame = single]
Fs=1e8;
Res = 100e3;
Nfft = Fs/Res;
[Psd_in,F_in] = pwelch(e0/sqrt(Fs),blackman(Nfft),
[],Nfft, Fs,'centered');

%setFigOptions
q = [200:800]; % plot only from 30MHz
figure;
plot(F_in(q)*1e-6, 10*log10(Psd_in(q)/max(Psd_in(q))),'k')
hold on

for i = 1:N
    [Psd_out(:,i),F_out(:,i)] = pwelch(e(i,:)/sqrt(Fs)
    ,blackman(Nfft),[],Nfft, Fs,'centered');
    plot(F_out(q,i)*1e-6, 10*log10(Psd_out(q,i)
    /max(Psd_out(q,i))))
end

grid on
xlabel('Frequency (MHz)');
ylabel('Normalised PSD (dB)');
legend('Ideal signal', '0.6 $\lambda$', '0.5 $\lambda$', 
'0.4 $\lambda$', '0.3 $\lambda$', '0.2 $\lambda$', 
'0.1 $\lambda$');
title('Power Spectral Density')

\end{lstlisting}

Code to plot ACPR
\begin{lstlisting}[frame = single]

for i = 1:N
    A=10*log10(Psd_out(q,i)/max(Psd_out(q,i)));
    Index = F_in(q); x = find(Index == 0); 
    IndexL = [x-150:x-60];
    IndexH = [x+60:x+150];
    Acpr_L=mean(A(IndexL));
    Acpr_H=mean(A(IndexH));

    ACPR(i) = (Acpr_L+Acpr_H)/2;
end

X = [0.6 0.5 0.4 0.3 0.2 0.1]';

figure(3)
plot(X,ACPR)
xlabel('Distance between antennas ($\lambda$)');
ylabel('ACPR (dB)')
title('ACPR vs spacing of Tx antennas')
ylim([-46 -37])

\end{lstlisting}



Code to plot antenna S-parameters
\begin{lstlisting}[frame = single]
%% Load S-parameters for two ant
clear
clc
close all

filename = ['SPAR/0p1.s2p'; 'SPAR/0p2.s2p'; 'SPAR/0p3.s2p'; 
'SPAR/0p4.s2p'; 'SPAR/0p5.s2p'; 'SPAR/0p6.s2p'];

for i = 1:6
file = fopen(filename(i,:));
Data = cell2mat(textscan(file,'%f %f %f %f %f %f %
f %f %f %*[^\n]','HeaderLines',5));
file = fclose(file);

f = Data(:,1);
S11 = Data(:,2)+1j*Data(:,3);
S11 = 10.^(S11/10);

S21 = Data(:,4)+1j*Data(:,5);
S21 = 10.^(S21/10);

S12 = Data(:,6)+1j*Data(:,7);
S12 = 10.^(S12/10);

S22 = Data(:,8)+1j*Data(:,9);
S22 = 10.^(S22/10);


plot(f,10*log10(abs(S22)))
hold on
end

title('S_{22}')
xlabel('Frequency (Hz)')
ylabel('S_{22} (dB)')

legend('d=0.1$\lambda$' ,'d=0.2$\lambda$' ,'d=0.3$\lambda$' 
,'d=0.4$\lambda$' ,'d=0.5$\lambda$' ,'d=0.6$\lambda$')
\end{lstlisting}


Code to plot antenna farfield
\begin{lstlisting}[frame = single]
%% plot antenna farfield
clear
clc
close all

file1 = fopen('Data/4th_element_4_array.txt');
Data1 = cell2mat(textscan(file1,'%f %f %f %f %f 
%f %f %f %*[^\n]','HeaderLines',2));
file1 = fclose(file1);

s = 121;
phi = Data1(72601:s:79860,2); %phi axis
theta = Data1(1:s,3); %theta axis

mag = zeros(length(phi),length(theta));
index = (72601:79860);
gaindB = Data1(index,8);

for i = 1:length(phi)
   mag(i,:) = gaindB((1+(s*(i-1)):(s*i)));
end
phi(end) = pi;

figure(1)
patternCustom(mag,theta*180/pi,phi*180/pi)

 gain = max(gaindB)
\end{lstlisting}







